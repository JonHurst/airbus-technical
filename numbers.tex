%% -*- outline-regexp: "%% [*]+"; -*-

\documentclass[twoside,10pt]{article}
\usepackage[T1]{fontenc}
\usepackage{txfonts}
\usepackage[paper=a5paper,inner=2cm,outer=1.25cm,tmargin=1.25cm,bmargin=1.25cm,includefoot]{geometry}
\usepackage{setspace}
\usepackage{graphics}
\usepackage{color}
\usepackage{paralist}
\usepackage{array}
\usepackage[bottom]{footmisc}
\pagestyle{plain}
\setlength{\parindent}{0pt}
\setlength{\parskip}{0em}
\title{Quick Reference\\\large Version 1.7.1}
\author{Jon Hurst}
\setcounter{secnumdepth}{0}


%% * Macros
\newcommand{\extref}[1]{{\color{blue}\scriptsize [#1]}}
\newcommand{\secref}[1]{\vspace{-3mm}\extref{#1}}

\newcommand{\mylist}[1]{
  \begin{list}{}{\setlength\parsep{0pt}\setlength\itemsep{0pt}}
    #1
\end{list}}

\newcommand{\mysec}[3][]{
  \vbox{
    \subsection{#2}
    #1
    \mylist{#3}
}}

\newcommand{\iitem}{\item\hspace{2em}}

\newcommand{\pitem}[1]{
\item\hfill\parbox{22em}{\begin{spacing}{0.9}\vspace{0.25em} #1 \vspace{0.8em}\end{spacing}}}

\newcommand{\rwstate}[1]{
  \parbox{0.5\textwidth}{
    \vspace{1mm}
    \begin{compactitem}
      #1
    \end{compactitem}
    \vspace{1mm}}}

%% * START
\begin{document}
\pagenumbering{roman}
\raggedbottom
\setcounter{tocdepth}{2}
\maketitle
\tableofcontents


% \pagebreak
% \vspace*{5cm}

% \begin{center}
%   Intentionally blank
% \end{center}
\pagebreak


\pagenumbering{arabic}
%% * Limitations
%% ** Wake Turbulence
\vbox{
\subsection{Wake Turbulence}
\secref{EOMA 8.3.9}
\subsubsection{Final Approach}
\begin{center}
  \begin{tabular}{|c|c|}\hline
    Type & Separation\\\hline
    A380 & 7nm\\
    Heavy & 5nm\\
    Upper Medium (e.g. 757,707) & 4nm (UK only)\\\hline
  \end{tabular}
\end{center}

\textbf{Note}: Boeing 757 and Boeing 737-800/900 are classified as heavy for the purposes of final
approach in some countries.

\subsubsection{Departure}
\begin{center}
  \begin{tabular}{|c|c|}\hline
    Type & Separation\\\hline
    A380 & 3 mins\\
    Heavy & 2 mins\\\hline
  \end{tabular}
\end{center}

\textbf{Note}: Add \textbf{1 minute} if departure is not from the same position
}

%% ** Contaminated runway ops
\mysec{Contaminated Runway Operations}{
\item Takeoff or landing is not permitted with: \extref{EOMB 2.1}
  \begin{compactitem}
  \item Wet ice
  \item Water on top of compacted snow
  \item Dry Snow or Wet Snow over Ice
  \item RWYCC 0
  \end{compactitem}

  \item Damp runway \extref{EOMB 4.5.1/4.12.1} \dotfill Use Wet performance
  \item Contamination $<$ 25\% of runway area \extref{EOMB 4.6.2} \dotfill
    Use Wet performance
  \item Contamination $<$3mm \extref{EOMB 4.6.2} \dotfill Use Wet performance
  \item Takeoff/Ldg Dispatch condition assessment matrix \dotfill OQA Ops Data
  \item Landing condition assessment matrix (RCAM)\dotfill OQA Ops Data
  \item Ldg Dispatch, wet ice and layered contaminants \extref{EOMB 2.1}\dotfill 2 alternates rqd
  \item Minimum cleared width \extref{EOMB 2.1} \dotfill 30m (check
    snowbanks: EOMB 4.6.8)\footnotemark
  \item ``Slippery when wet''
    \iitem Takeoff \dotfill EOMB 4.5.4
    \iitem Landing \dotfill EOMB 4.12.3
}
\footnotetext{Cleared width $\leq$45m does not need to be treated as a narrow runway for V$_1$ etc.}
%% ** Wind limits
\vbox{
  \subsection{Takeoff and Landing Wind Limits}
  \subsubsection{Takeoff or Manual Landing, Runway Width $\geq$45m}
  \secref{FCOM LIM.AG.OPS}
  \mylist{
  \item Max Headwind\dotfill No limit
  \item Max Tailwind\dotfill 15kt (landing: Conf Full if >10kt)
  }

  \begin{center}
  \begin{tabular}{|l|c|}\hline
  \multicolumn{1}{|c|}{Runway Condition}
  & Max Crosswind\\  ~ & (including gust)\\\hline
  \rwstate{
    \item Dry
    \item Damp
    \item Wet ($\leq$3mm water)
    \item Slush ($\leq$3mm)
    \item Snow ($\leq$3mm, dry or wet)
    \item Frost
    \item $\leq$25\% contamination} &
  \parbox{0.35\textwidth}{
    \vspace{1mm}
    \begin{compactitem}[]
    \item NEO Takeoff\dotfill 35kt
    \item Otherwise\dotfill 38kt
    \end{compactitem}
    \vspace{1mm}
  }\\\hline

  \rwstate{\item Compacted Snow (OAT$\leq$-15°C)} & 29kt\\\hline

  \rwstate{
  \item Dry Snow (>3mm, $\leq$100mm)
  \item Wet Snow (>3mm, $\leq$30mm)
  \item Compacted Snow (OAT>-15°C)
  \item Dry Snow over Compacted Snow
  \item Wet Snow over Compacted Snow
  \item Slippery when wet} & 25kt\\\hline

  \rwstate{
  \item Water (>3mm, $\leq$12.7mm)
  \item Slush (>3mm, $\leq$12.7mm)} &20kt\\\hline

  \rwstate{\item Ice (cold and dry)} & 15kt\\\hline

  \end{tabular}
  \end{center}
}

\vbox{
  \label{autoland_wind_limits}
  \subsubsection{Autoland without automatic rollout}
  \secref{FCOM LIM.AFS.20}
  \mylist{
  \item Headwind
    \iitem A321\dotfill 15kt
    \iitem A319 OEI\dotfill 15kt
    \iitem A319 AEO\dotfill 20kt
    \iitem Otherwise\dotfill 30kt
  \item Tailwind
    \iitem A319 CONF3\dotfill 5kt
    \iitem Otherwise\dotfill 10kt
  \item Crosswind
    \iitem A321\dotfill 10kt
    \iitem A319 OEI\dotfill 10kt
    \iitem Otherwise\dotfill 20kt
    \iitem Crosswind must not be greater than that for manual landing.
  }}

  \vbox{
  \subsubsection{Autoland with automatic rollout}
  \secref{FCOM LIM.AFS.20}
  \mylist{
  \item Headwind
    \iitem A321\dotfill 15kt
    \iitem A319 OEI\dotfill 15kt
    \iitem A319 AEO\dotfill 20kt
    \iitem Otherwise\dotfill 30kt
  \item Tailwind
    \iitem A319 CONF3\dotfill 5kt
    \iitem Otherwise\dotfill 10kt
  \item Crosswind
    \iitem A321\dotfill 10kt
    \iitem A319 OEI\dotfill 10kt
    \iitem A320 CEO with sharklets\dotfill 15kt
    \iitem A320 NEO with inop thrust reverser\dotfill 15kt
    \iitem\hfill (idle reverse used on remaining reverser)
    \iitem Otherwise\dotfill 20kt
  }}

\vbox{
  \subsubsection{Takeoff or Manual Landing, Runway Width $<$45m, $\ge$30m}
  \secref{FCOM PRO.SPO.60}
  \mylist{
  \item Max Headwind\dotfill No limit
  \item Max Tailwind\dotfill 15kt (Conf Full if >10kt)
  }
  \begin{center}
  \begin{tabular}{|l|c|}\hline
  \multicolumn{1}{|c|}{Runway Condition}
  & Max Crosswind\\  ~ & (including gust)\\\hline

  \rwstate{\item Dry} &
  \parbox{0.35\textwidth}{
    \vspace{1mm}
    \begin{compactitem}[]
    \item NEO Takeoff\dotfill 35kt
    \item Otherwise\dotfill 38kt
    \end{compactitem}
    \vspace{1mm}}\\\hline

  \rwstate{\item Wet ($\leq$3mm water)} & 33kt\\\hline

  \rwstate{\item Contaminated (not icy)} & 10kt\\\hline

  \end{tabular}
  \end{center}
}


%% ** Other Autoland limits
\mysec[\secref{FCOM LIM.AFS.20}]{Other Autoland limitations}{
\item Alert height \dotfill 100ft
\item Approved configurations
  \iitem A320CEO OEI\dotfill CONF FULL only
  \iitem Otherwise\dotfill CONF 3, CONF Full
\item Rwy conditions for automatic rollout \dotfill Dry, Wet
\item Glideslope \dotfill CEO:2.5° to 3.15°; NEO: 2.5° to 3.25°
\item Max airport elevation
  \iitem A320CEO\dotfill2500ft
  \iitem A319, A320NEO, A321NEO\dotfill9200ft
\item Min pressure altitude\dotfill NEO:-2000ft; CEO:-1000ft
\item Auto rollout with one reverser inop\dotfill Idle
  reverse only\footnotemark
\item A319 Max weight (emergency only)\dotfill 69000kg
\item Min weight
  \iitem A320 NEO\dotfill44000kg
  \iitem A321 NEO\dotfill52500kg
}\label{other_autoland_limits}
\footnotetext{no limitation for A320CEO without sharklets}

%% ** Other wind limits
  \mysec{Other wind limits}{
    \item LEAP-1A starting max crosswind \extref{FCOM LIM.EBG}\dotfill 45kt
\item Passenger door operation \extref{FCOM LIM.AG.OPS}\dotfill 65kt
\item Cargo door operation \extref{FCOM LIM.AG.OPS}\dotfill 40kt (50kt w/caveats)
\item Cargo door closed before \extref{FCOM LIM.AG.OPS}\dotfill 65kt
}

%% ** First Officer limits
\mysec[\secref{EOMB 2.1}]{First Officer limits}{
\item 3* FO\dotfill No planned tailwind, no flap 3 landing
\item Max crosswind\dotfill 20kt
\item Takeoff minimum\dotfill 550m RVR
\item Circling minima\dotfill 5000m
\item Min runway width, no specific training\dotfill 45m
\item No contaminated, slippery, RWYCC$\le$4, windshear or autoland
}


%% ** Airport limits
\mysec{Airport limitations}{
\item Max slope \extref{FCOM LIM.AG.OPS} \dotfill $\pm$2\%
\item Max runway altitude \extref{FCOM LIM.AG.OPS} \dotfill 9200ft
\item Nominal runway width \extref{FCOM LIM.AG.OPS} \dotfill 45m
\item Minimum runway width \extref{FCOM LIM.AG.OPS} \dotfill 30m
\item Min planning fire fighting category \extref{EOMA 8.1.2.1}
  \iitem Departure/Destination  \dotfill A321:7, A319/A320: 6\footnotemark
  \iitem Alternates \dotfill UK: 5; Non-UK: 4
}
\footnotetext{Up to 2 categories less is OK with caveats -- see EOM-A}

\pagebreak
%% ** Dimensons
\vbox{
  \subsection{Aircraft Dimensions}
  \secref{FCOM DSC.20.20}
\begin{center}
\begin{tabular}{|l|c|c|c|c|}\hline
  & \parbox{.15\textwidth}{\centering A319}
  & \parbox{.15\textwidth}{\centering A320 \small(no sharklets)}
  & \parbox{.15\textwidth}{\centering A320 \small(sharklets)}
  & \parbox{.15\textwidth}{\centering A321} \\\hline
  Wingspan  & \multicolumn{2}{c|}{34.1m} & \multicolumn{2}{c|}{35.8m}\\\cline{2-5}
  Length  & 33.84m & \multicolumn{2}{c|}{37.57m} & 44.6m\\\cline{2-5}
  180° turn  & 20.5m & \multicolumn{2}{c|}{22.8m} & 28.3m\\\cline{2-5}
  Widest sweep  & \multicolumn{3}{c|}{Wing tip} & Tail \\\hline
\end{tabular}
\end{center}}

%% ** Weight limits
\vbox{
  \subsection{Weight Limits}
  \secref{FCOM LIM.AG.WGHT, GHM E.4}
\begin{center}
\begin{tabular}{|l|c|c|c|c|}\hline
  Weight & A319 & A320CEO & A320NEO & A321 \\ \hline
  Max Takeoff & 68000kg & 77000kg & 79000kg & 89000kg\\
  Max Taxi & 68400kg & 77400kg & 79400kg & 89400kg\\
  Max Landing & 61000kg & 66000kg & 66300kg & 77300kg\\
  Max Zero Fuel & 57000kg & 62500kg & 62800kg & 73300kg\\
  Minimum & 35400kg & 37230kg & 40600kg & 46600kg\\ \hline
\end{tabular}
\vspace*{1.5mm}\\
\begin{tabular}{|l|c|c|l|c|c|}\cline{1-2}\cline{4-6}
& A319 & & & A320 & A321\\\cline{1-2}\cline{4-6}
Compartment 1 & 2268kg & & Compartment 1 & 3402kg & 2202kg\\
& & & Compartment 2 &  & 3468kg\\
Section 41 & 1326kg & & Compartment 3 & 2426kg & 3187kg\\
Section 42 & 1695kg & & Compartment 4 & 2110kg & 1683kg\\
Compartment 5 & 1497kg & & Compartment 5 & 500kg & 500kg\\\cline{1-2}\cline{4-6}
\end{tabular}
\end{center}
}

%% ** Passenger zones
\vbox{
  \subsection{Passenger zones}

  \secref{EOMB 7.3.2}

  \begin{center}
  \begin{tabular}{|c||c|c|c|}
    \hline
      & A & B & C\\
      \hline
      A319 & 1-9 & 10-18 & 19-26\\
      A320 & 1-10 & 11-20 & 21-31\\
      A321 & 1-13 & 14-28 & 29-40\\\hline
  \end{tabular}
  \end{center}

    Zone counts must be accurate to within ±3 passengers.
  }

%% ** Loading
\mysec{Loading}{
  \item \textbf{A319 Protocol:}\extref{GHM 5.4.1.2}
    \iitem First 150 bags to Section 41/42
    \iitem Next 50 to Compartment 5
    \iitem Overspill to Compartment 1

  \item\textbf{A320 Protocol:}\extref{GHM 5.4.1.2}
    \iitem First 85 bags to Compartment 1
    \iitem Next 60 bags to Compartment 3
    \iitem Overspill to Compartment 4
    \iitem No planned usage of Compartment 5

  \item\textbf{A321 Protocol:}\extref{GHM 5.4.1.2}
    \iitem First 100 bags to Compartment 3
    \iitem Next 50 bags to Compartment 4
    \iitem Overspill to Compartment 2

\item LMCs \extref{EOMB.7.4}
\iitem $\Delta$Weight $\le$ +250kg \textbf{and} |$\Delta$CG| $\le$ 2\% \dotfill No action required
\iitem $\Delta$Weight > +250kg \textbf{or} |$\Delta$CG| > 2\% \dotfill Recalculate performance
\iitem New paperwork required for -20/+10 passengers
%\item A320 Standard CG \extref{EOMB 4.9.4} \dotfill CG > 27\%MAC
\item A320 Forward CG \extref{EOMB 4.9.1.1} \dotfill CG < 27\%MAC
}

\pagebreak

%% ** Turbulence speeds
\mysec[\secref{FCOM PRO.ABN.MISC}]{Turbulence speeds}{
\item A319/A320CEO
  \iitem $<$FL200\dotfill 250kt
  \iitem $\ge$FL200\dotfill 275kt
  \iitem $\ge$FL320\dotfill M0.76
\item A320NEO
  \iitem $<$FL200\dotfill 260kt
  \iitem $\ge$FL200\dotfill 280kt
  \iitem $\ge$FL310\dotfill M0.76
\item A321NEO
  \iitem $<$FL200\dotfill 275kt
  \iitem $\ge$FL200\dotfill 305kt
  \iitem $\ge$FL270\dotfill M0.76
}
%% ** General speeds

\mysec[\secref{FCOM LIM.AG.SPD}]{General speeds}{
\item V$_{MO}$ \dotfill 350kt
\item M$_{MO}$ \dotfill 0.82M
\item Max Tire speed \dotfill 195kt
\item Max speed for wipers \dotfill 230kt
\item Max speed cockpit window open \dotfill 200kt
\item V$_{MCA}$ (rounded up)
  \iitem A319 \dotfill 0ft:108kt; 2000ft:106kt
  \iitem A320CEO \dotfill 0ft:110kt; 2000ft:108kt
  \iitem A320NEO \dotfill 0ft:114kt; 2000ft:114kt
  \iitem A321NEO \dotfill 0ft:110kt; 2000ft:107kt
\item V$_{MCG}$ CONF 1+F (rounded up)
  \iitem A319 \dotfill 0ft:105kt; 2000ft:103kt
  \iitem A320CEO \dotfill 0ft:111kt; 2000ft:109kt
  \iitem A320NEO \dotfill 0ft:116kt; 2000ft:116kt
  \iitem A321NEO \dotfill 0ft:118kt; 2000ft:115kt
}

%% ** Engine limitations
\mysec[\secref{FCOM LIM.ENG}]{Engine}{
\item TOGA time \dotfill 5 mins (10 mins single engine)
\item EGT
\iitem TOGA \dotfill CFM56:950°C; LEAP-1A:1060°C
\iitem MCT \dotfill CFM56:915°C; LEAP-1A:1025°C
\iitem CFM56 Start \dotfill 725°C
\iitem LEAP-1A Air Starting\dotfill 875°C
\iitem LEAP-1A Ground Starting\dotfill 750°C
\item Oil
  \iitem CFM56 min quantity\extref{EOMB 2.3.4.8}\dotfill 9.5qt+0.5qt/hr
  \iitem LEAP-1A min quantity\extref{EOMB 2.3.4.8}\dotfill 8.9+0.45qt/hr; minimum 10.6qt\footnotemark

\iitem Max cont temp \dotfill 140°C
\iitem Max trans temp \dotfill 155°C
\iitem Min start temp \dotfill -40°C
\iitem Min takeoff temp \dotfill CFM56:-10°C; LEAP-1A:19°C
\iitem LEAP-1A Min Oil Pressure\dotfill 17.4psi
\iitem LEAP-1A Max Oil Pressure (oil temp < 50°C)\dotfill 130.5psi
\iitem LEAP-1A Max Oil Pressure (oil temp > 50°C)\dotfill 145psi

\item Max N1 \dotfill CFM56:104\%; LEAP-1A:101\%
\item Max N2 \dotfill CFM56:105\%; LEAP-1A:116.5\%
\item CFM56 Starter\footnotemark
  \iitem No running engagement when N2>20\%
  \iitem Pause between cycles\dotfill 20 sec
  \iitem Cooling period after 4 failed cycles\dotfill 15min
\item LEAP-1A Starter\footnotemark[\value{footnote}]
  \iitem No running engagement when N2>63\%
  \iitem Pause between cycles\dotfill 60 sec
  \iitem Cooling period after 3 failed cycles\dotfill 15min
\item Max reverse \dotfill >70kt
}
\footnotetext[\numexpr\value{footnote}-1]{If engines have been shut down for >60min, decrease requirement by 3qt}
\footnotetext{An automatic start that includes up to three start attempts is considered one cycle}

%% ** Fuel limitaions
\mysec[\secref{FCOM LIM.FUEL}]{Fuel Limitations}{
\item Min qty for takeoff \dotfill 1500kg

  \subsubsection{Temperature}
\item Jet A1, minimum \dotfill -43°C
\item Jet A1, maximum \dotfill CEO:54°C; NEO:55°C
  \subsubsection{A319/A320 fuel imbalance}
\item Takeoff
  \iitem Outer tanks balanced\dotfill 500kg\footnotemark
  \iitem Inner tanks balanced\dotfill 370kg

\item In-flight, outer tanks balanced
\iitem One inner tank full\dotfill 1500kg
\iitem Fuller inner tank 4300kg\dotfill 1600kg
\iitem Fuller inner tank $<$ 2250kg\dotfill No Limitation
\item In-flight, max outer tank imbalance OK if sides are balanced\footnotemark
  \subsubsection{A321 fuel imbalance}
\item Takeoff\dotfill 400kg\footnotemark
\item In-flight
  \iitem One wing tank full\dotfill 1320kg
  \iitem Fuller wing tank 4000kg\dotfill 1450kg
  \iitem Fuller wing tank $\leq$2350kg\dotfill No limitation
}

\footnotetext[\numexpr\value{footnote}-2]{This is the figure for the heavier
  inner tank being full, which is the worst case. Check the FCOM if this is too
  limiting.}

\footnotetext[\numexpr\value{footnote}-1]{Also OK if lighter outer tank, and the heavier inner tank are on
  the same side, with the difference between the inner tanks $\leq$3000kg.}

\footnotetext{This is the figure for the worst case scenario where the heavier wing tank has
  $\geq$3000kg. Check the FCOM if this is too limiting.}

%% ** Fuel capacity
\mysec[\secref{FCOM DSC.28.10.20}]{Fuel Capacity}{
\item A319/A320 (approximate -- varies slightly between airframes)
\iitem Outer tanks \dotfill 2 x 700kg
\iitem Inner tanks \dotfill 2 x 5500kg
\iitem Center tank \dotfill 6500kg
\iitem Total \dotfill 18900kg
\item A321
\iitem Wing tanks \dotfill 2 x 6000kg
\iitem Center tank \dotfill 6500kg
\iitem Total \dotfill 18500kg
}

%% ** Hydraulics limitations
\mysec[\secref{FCOM DSC.29.10}]{Hydraulics}{
\item Normal pressure\dotfill 3000psi
\item RAT only pressure\dotfill 2500psi
\item Green/Yellow differential for PTU activation\dotfill 500psi
}

%% ** Electrical limitations
\mysec{Electrical}{
\item Prelim prep min battery voltage \extref{EOMB 2.3.4.2} \dotfill $>$25.5V
\item Battery check \extref{EOMB 2.3.6.2} \dotfill charge current<60A within 10s
}

%% ** Pressurisation limitations
\mysec[\secref{FCOM LIM.AIR}]{Pressurisation}{
\item Max pos diff \dotfill 9.0psi
\item Max neg diff \dotfill -1psi
\item Safety valve \dotfill 8.6psi
\item Max norm cabin alt \dotfill 8000ft
\item Cab alt warning \dotfill 9550ft$\pm$350ft
\item Ram air max diff \extref{FCOM DSC.12.10.20}\dotfill 1psi
}

%% ** Air conditioning / Ventilation limitations
\mysec[\secref{FCOM LIM.AIR, FCOM PRO.NOR.SUP.ADVWXR}]{Air conditioning / Ventilation}{
  \item Passengers on board without Air Con\dotfill Max 20mins
\item Do not simultaneously use packs and ground LP Air Conditioning Unit.
\item Do not use HP ground unit when APU is supplying bleed air.
\item Max OAT for norm avionics ventilation \dotfill 49°C\footnotemark
\item Max OAT with EXTRACT OVRD and Packs Off \dotfill  39°C\footnotemark[\value{footnote}]
}
\footnotetext{Higher temperatures are allowable with a time limit -- see FCOM}

%% ** Ice protection limitations
\mysec{Ice protection}{
\item Ground icing conditions \extref{EOMB 2.3.9}\dotfill OAT$\le$10°C + vis moist | gnd contam
\item Flight icing conditions \extref{EOMB 2.3.9}\dotfill TAT$\le$10°C + vis moist
\item Eng anti-ice not rqd \extref{EOMB 2.3.13}\dotfill climb/cruise,
  SAT<-40°C, no CBs
  \subsubsection{Accreted ice \secref{FCOM PRO.NOR.SUP.ADVWXR}}
  \item Wing anti-ice operative
  \iitem Min speed, Conf Full \dotfill V$_{\mathrm{LS}}$ + 5kt
  \iitem Min Speed, < Conf Full \dotfill V$_{\mathrm{LS}}$ + 10kt
\item Wing anti-ice inoperative
  \iitem Min speed\dotfill V$_{\mathrm{LS}}$ + 10kt/ GREEN DOT
\item Avoid extended flight in icing conditions with slats extended
\item Use ``Ice Accretion'' in A-ICE field of FS+
  \subsubsection{Fan ice shedding \secref{EOMB 2.3.9}}
\item CFM56
  \iitem Icing conditions, OAT$\leq$3° \dotfill 70\%N1 for 30 secs every 30 mins
  \iitem FZRN/FZDZ/FZFG/+SN \dotfill 70\%N1 every 10 mins, no hold time
  \iitem Before takeoff \dotfill 70\%N1 for 30 secs
\item LEAP-1A
  \iitem Icing conditions, OAT$\leq$3° | eng vib \dotfill 50\%N1 for 5 secs every 60 mins
  \iitem Icing conditions on ground > 120 mins\dotfill Inspection rqd
  \iitem Before takeoff \dotfill 50\%N1 for 5 secs
}


%% ** Flaps/slats limitations
\mysec[\secref{FCOM LIM.F\_CTL, FCOM LIM.AG.SPD}]{Flaps/slats}{
\item Max flaps/slats altitude \dotfill 20000ft

  \subsubsection{A319/A320}
\item Conf 1 \dotfill 230kt
\item Conf 1+F \dotfill 215kt
\item Conf 2 \dotfill 200kt
\item Conf 3 \dotfill 185kt
\item Conf Full \dotfill 177kt

  \subsubsection{A321}
\item Conf 1 \dotfill 243kt
\item Conf 1+F \dotfill 225kt
\item Conf 2 \dotfill 215kt
\item Conf 3 \dotfill 195kt
\item Conf Full \dotfill 186kt
}

%% ** Gear limitations
\mysec{Gear}{
\item Extend \extref{LIM.AG.SPD}\dotfill 250kt
\item Retract \extref{LIM.AG.SPD}\dotfill 220kt
\item Extended \extref{LIM.AG.SPD}\dotfill 280kt/M.67
\item Max taxi speed, single tyre deflated \extref{LIM.LG}\dotfill 7kt
\item Max taxi speed, both tyres deflated \extref{LIM.LG}\dotfill 3kt
\item Max steering angle, both tyres deflated \extref{LIM.LG}\dotfill 30°
\item Max brake temp for takeoff \extref{LIM.LG}\dotfill 300°
}

%% ** APU limitations
\mysec[\secref{FCOM LIM.APU}]{APU}{
\item Starter duty  \dotfill 3 cycles then 60 mins
\item Maximum N \dotfill 107\%
\item Max start EGT  \dotfill <35000ft: 1090°C, >35000ft 1120°C
\item Max running EGT  \dotfill 675°C
\item Max Altitudes
\iitem Two packs \dotfill $<$15000ft
\iitem Engine start \dotfill $<$20000ft
\iitem One pack \dotfill $<$22500ft
\iitem Electrical power \dotfill $<$41000ft
\iitem Battery start (emerg elec config) \dotfill $<$25000ft
\iitem Normal start \dotfill $<$41000ft
\iitem  Air bleed for wing anti-icing \dotfill Not permitted
\item Approximate fuel burn \extref{EOMB 5.1} \dotfill 2kg/min
\item Ops with ``Low Oil Level'' ECAM \extref{FCOM.PRO.ABN.ADV} \dotfill 10hrs
}

%% ** Navigation limitations
\mysec{Navigation}{
\item Max IRS latitudes \extref{FCOM LIM.NAV}\dotfill 73°N,60°S\footnotemark
\item Altimeter tolerances \extref{EOMB 2.3.6.11}
\iitem ADR vs Airfield elevation\dotfill $\pm$75ft
\iitem ADR vs ADR\dotfill $\pm$20ft
\iitem ISIS vs ADR\dotfill $\pm$100ft
\item Altimeter temperature corrections \dotfill QRH SI
}
\footnotetext{This is worst case assuming all ADIRUs have the same magnetic variation table. Some
  aircraft are able to fly to 82°N for some longitudes. If one ADIRU has a different table check the
  FCOM for more limiting maximum latitudes.}
%% ** Oxygen limitations
\mysec[\secref{FCOM LIM.OXY}]{Oxygen}{
  \subsubsection{Minimum Dispatch Oxygen Pressure}
\item A319/A320CEO, ref temp 40°\footnotemark
\iitem 3 crew\dotfill 1024psi
\iitem 2 crew\dotfill 781psi
\item A320NEO/A321NEO ref temp 40°\footnotemark[\value{footnote}]
\iitem 3 crew\dotfill CAPT:780psi, FO:550psi
\iitem 2 crew\dotfill CAPT:550psi, FO:550psi
\subsubsection{Minimum Endurance}
\item Emergency Descent (reg normal), 3 crew
  \iitem A319/A320CEO\dotfill 13 mins
  \iitem A320NEO/A321NEO\dotfill 15mins
\item Cruise at FL100 (reg normal), 2 crew
  \iitem A319/A320CEO\dotfill 107mins
  \iitem A320NEO/A321NEO\dotfill 105mins
\item Fire (reg 100\%), 8000ft, all crew\dotfill 15mins
}
\footnotetext{On ground, ref temp is average of OAT and cockpit temp.}
%% ** Autopilot limitations
\mysec[\secref{LIM.AFS}]{Autopilot}{
\item Engagement after TO \dotfill $>$100ft agl and $>$5 secs
\item Engagement after manual go-around \dotfill $>$100ft agl
\item Minimum approach engagement height
\iitem Circling \dotfill 500ft agl
\iitem RNAV visual approach \dotfill 500ft agl
\iitem FINAL APP, V/S or FPA mode \dotfill 250ft agl
\iitem PAR \dotfill 250ft agl
\iitem ILS, CAT1 displayed on FMA \dotfill 160ft agl
\iitem SBAS, APPR1 displayed of FMA \dotfill 160ft agl
\iitem Cat II approach, manual landing \extref{FCOM LIM.AFS.20} \dotfill 80ft agl
\iitem
\iitem Other phases A319/A320 \dotfill 500ft agl
\iitem Other phases A321 \dotfill 900ft agl
}

%% ** Autoland warning light
\mysec[\secref{DSC.22.30.80.30}]{Autoland Warning Light}{
\item[$\bullet$] RA<200ft and
\item[$\bullet$] one or more autopilots engaged and
\item[$\bullet$] LAND or FLARE annunciated and
\item[$\bullet$] one or more of:
  \iitem Both autopilots disconnect (N.B. RA NCD case)
  \iitem RA>15ft and localiser signal lost or deviation >¼ dot
  \iitem RA>100ft and glideslope signal lost or deviation >1 dot
  \iitem Difference between rad alt > 15ft
  \iitem Long or untimely flare detected (newer aircraft only)
}
%% ** Airport lighting
\mysec{Airport lighting}{
\item Runway lighting \extref{LIDO GEN-1.3.1.6}
\iitem Red and white \dotfill 900m
\iitem Red \dotfill 300m
}

%% ** Miscellaneous
\mysec{Miscellaneous}{
\item German corner \dotfill KRH R270/12D
\item Alternate ranges \extref{EOMB 5.1}
\iitem Takeoff \dotfill 320nm
\iitem Enroute \dotfill A319:380nm; A320/A321:400nm
\item Ballpark diversion fuel \dotfill 15kg/nm
\item Approx Power settings
\iitem Two engine approach, V$_{app}$ \dotfill 50\% N1
\iitem Single engine approach, V$_{app}$ \dotfill 70\% N1
\iitem Two engine cruise \dotfill (50 + Altitude/1000)\% N1
\item Flex corrections \extref{EOMB 2.3.10}
\iitem Anti-ice on \dotfill subtract 5°C\footnotemark
\iitem QNH reduction \dotfill subtract 1°C/2hPa\footnotemark[\value{footnote}]

\item easyJet Landing Distance Factor \dotfill 1.15
\item Rev Idle on Wet runway OK\dotfill RWYCC 2, Unfactored LD < LDA
}
\footnotetext{Flex must remain >TREF (A319:ISA+30; A320:ISA+29; A321:ISA+15) and >OAT}
\pagebreak

%% * Engine failure
%% ** Single engine operations
\mysec{Single Engine operations}{
\item Avoid reducing below V$_\mathrm{LS}$ (including GPWS/Windshear) \extref{FCOM PRO.ABN.ENG}
\item Autoland capability \extref{FCOM LIM.AFS.20}\dotfill Cat III single (A320CEO $\Rightarrow$ Conf
  Full)
\item Available NPA Autopilot modes \extref{FCOM LIM.AFS.10}
  \iitem A320/A321 \dotfill All
  \iitem A319 \dotfill LOC/VS, LOC/FPA, HDG/VS, TRK/FPA\footnotemark
\item Do not extend full flaps until established on final descent. \extref{FCOM PRO.ABN.ENG}
\item Use Conf 3 if a level off is required. \extref{FCOM PRO.ABN.ENG}
\item Check QRH ABN.ENG.OEI Circling Approach if circling required
\item Sharklet automatic rollout \extref{FCOM LIM.AFS} \dotfill Idle reverse only
}
\footnotetext{All modes are permitted Flight Director only}

%% ** Dual engine failure
\mysec[\secref{QRH ABN.ENG}]{Double engine failure}{
\item Landing Configuration
  \iitem Forced Landing\dotfill Flap 2 (slats only), Gear Down
  (gravity extension)
  \iitem Ditching\dotfill Flap 2 (slats only), Gear Up, Ditching
  button pushed
\item Still air glide ratios
  \iitem 300kt \dotfill 2nm per 1000ft (500ft per nm, 4.7°)
  \iitem Green dot \dotfill 2½nm per 1000ft (400ft per nm, 3.75°)
  \iitem CONF 2, Gear Down \dotfill 1.6nm per 1000ft ($\sim$600ft per nm, 5.6°)
\item Headwind correction \dotfill $\sim$50ft per nm for each 10kt average headwind
\item Altitude loss in turn \dotfill $\sim$1000ft per 90° (conservative at low altitudes)
}

%% * Emergency calls
\mysec[\secref{EOMB 3.80.5}]{Emergency calls to cabin}{
\item Ground ops alert:
  \pitem{``Attention! Crew at Stations''}
\item Notification of a potential emergency in-flight:
  \pitem{``Attention! Crew at Stations''}
\item Alert cancellation :
  \pitem{``Cabin crew, normal operations''}
\item Evacuation :
  \pitem{``Evacuate. Unfasten your seatbelts and get out''}
\item NITS on flight deck :
  \pitem{``Senior cabin crew member to the flight deck''}
\item NITS via interphone :
  \pitem{``Senior cabin crew member to the interphone'' or 3 double chimes}
\item Unplanned emergency landing :
  \pitem{``Attention, Crew! Brace, Brace!''}
\item Rapid decompression \extref{EOMB 3.80.2}:
  \pitem{Auto announcement -- no Flight Crew PA required.}
\item Planned emergency landing :
  \iitem 2000ft\dotfill ``Cabin crew, take up landing positions''
  \iitem 500ft\dotfill ``Brace, Brace''
\item Severe turbulence :
  \pitem{``Cabin crew and passengers be seated immediately''}
}

\pagebreak
%% * Procedures
\section{Procedures}

%% ** Tailwind / strong crosswind takeoff
\mysec[\secref{EOMB 2.3.12}]{Takeoff with tailwind or crosswind > 20kt}{
  \item $\bullet$ Set 50\% N1
  \item $\bullet$ Release brakes
  \item $\bullet$ Set 70\% N1
  \item $\bullet$ When GS>15kt increase thrust to reach TO thrust by 40kt GS.
  \item $\bullet$ Stick full forward until 80kt, neutral by 100kt
}

%% ** General
\mysec{General}{
\item Procedure turn \extref{LIDO GEN.1.5.5.6.1.2} \dotfill 75 seconds from start of 45° turn
}


%% ** Holding
\mysec[\secref{LIDO GEN.1.5.5.7}]{Holding}{
\item Joining
\iitem Turn anti-holdwise<110° to Outbound track \dotfill Parallel entry
\iitem Turn holdwise<70° to Outbound track \dotfill Teardrop entry
\iitem Otherwise \dotfill Direct entry
\item Standard\dotfill Right turns, parallel to airway
\iitem $\le$14000\dotfill 1 minute
\iitem >14000\dotfill 1½ minutes
\item Max holding speeds (Normal(Turbulent))
\iitem $\le$14000\dotfill 230kt(280kt)
\iitem >14000, $\le$20000\dotfill 240kt(280kt/0.8M)
\iitem >20000, $\le$34000\dotfill 265kt(280kt/0.8M)
\iitem >34000\dotfill 0.83M
}

%% ** PRNAV SID/STAR
\mysec[\secref{FCOM PRO.SPO.51}]{PRNAV (RNP 1) SID/STAR}{
\item[$\bullet$] \textbf{Minimum required equipment:}\\
  1 FMGC; 1 MCDU; 1 GPS or 2 DMEs; 2 IRS; 1 FD; PFD and ND on PF side; 1
  EFIS display on PM side. Additional procedure specific restrictions may
  be published. Check RNP 1.
\item[$\bullet$] If GNSS is used as primary navigation source, check
  RAIM availability. If GNSS is not required and GPS PRIMARY is not
  available, carry out a navigational reasonableness check prior to
  IAWP.
  \item[$\bullet$] Database procedure should not be changed except for
    the addition of missing altitude or speed constraints. ATC ``direct
    to'' instructions may be accepted when above MSA. Max allowable XTK Error on
    RF\footnotemark[1] leg is 0.5nm.\extref{EJ ppt}
\item[$\bullet$] If GPS PRIMARY LOST or NAV ACCUR DOWNGRAD on one ND or MCDU
  continue with the unaffected one.
\item[$\bullet$] Use raw data to identify and continue with FMGC that
  provides the correct position in case of:
  \iitem GPS PRIMARY lost on both NDs/MCDUs
  \iitem FMS1/FMS2 POS DIFF message
  \iitem CHECK IRS 1(2)(3)/FM POSITION (on MCDU)
  \iitem CHECK A/C POSITION
  \iitem FM/GPS POS DISAGREE ECAM
\item[$\bullet$] Request reclearance if:
  \iitem NAV ACCUR DOWNGRAD on both sides
  \iitem Additional restriction (e.g. Dual FMC, GPS) no longer met
}
\footnotetext[1]{Radius to Fix}


%% ** Visual Approach
\pagebreak
\mysec[\secref{EOMB 2.3.18.3.7}]{Visual Approach}{
\item[$\bullet$] Minimas: \extref{EOMA 8.1.3.6}
  \iitem Visibility \dotfill 5km
  \iitem Cloud ceiling \dotfill 2500ft aal
\item[$\bullet$] Downwind at Flap 1, S speed
\item[$\bullet$] Flap 2, F speed at start of base turn
\item[$\bullet$] Base Turn timing: 3s/100ft $\pm$1s/kt of hw/tw from abeam
  threshold
\item[$\bullet$] Shortly after turning base, gear down, continuous
  descent to landing.
}

%% ** Circling
\mysec[\secref{EOMB 2.3.18.3.5}]{Circling}{
\item[$\bullet$] If OEI, check weight (QRH ABN.19)
\item[$\bullet$] Protected circling radius \extref{LIDO GEN.1.5.5.6.2.5} \dotfill
  4.2nm from all RWY THR
\item[$\bullet$] Landing runway in secondary flight plan
\item[$\bullet$] Initial approach: Conf 3, gear down, F speed
\item[$\bullet$] Select TRK/FPA
\item[$\bullet$] Push V/S at least 100ft above circling minima
\item[$\bullet$] Turn 45°
\item[$\bullet$] 30 secs from wings level turn downwind (gives
  runway offset $\approx$1.7nm)
\item[$\bullet$] Activate secondary when downwind
\item[$\bullet$] Descent point 3 secs per 100ft past abeam threshold
\item[$\bullet$] Full flap when turning finals
}

%% ** Non-precision approach
\mysec[\secref{EOMB 2.3.18.3.4}]{Non-Precision Approach - Selected vertical and lateral}{
\item[$\bullet$] Fly a stabilised approach
\item[$\bullet$] 0.3nm before descent point, set and pull FPA
\item[$\bullet$] 1°FPA modifies descent profile by 100ft for each nm
}

%% ** RNAV / Overlay approach
\mysec[\secref{EOMB 2.3.18.3.2/4}]{RNAV or Non-Precision overlay approach}{
\item[$\bullet$] \textbf{Minimum required equipment:}\\ 1 FMS; 1 GPS; 2 IRS;
  1 MCDU; 1 FD; 1 PFD (on PF side); 2 ND (temporary display OK); 2 FCU channels. \extref{PRO.SPO.51}
  \item[$\bullet$] Max acceptable difference between altimeters \dotfill 100ft
\item[$\bullet$] RNP(AR)\footnotemark approaches are not yet authorised. \extref{EOMA 8.3.2.6}
%% \item[$\bullet$] Check RAIM availability for ETA$\pm$15 mins. \extref{EOMA 8.4.5.2}
%% \item[$\bullet$] Conventional approach must be available at destination
%%   or alternate. \extref{EOMA 8.4.5.2}
\item[$\bullet$] A minimum procedure temperature will be promulgated on
  the approach chart:

  \iitem $\bullet$ Above this temp \dotfill fully managed, uncorrected VNAV DA
  \iitem $\bullet$ Below this temp \dotfill corrected gp, NAV/FPA, corrected LNAV DA
\item[$\bullet$] Modifications to FMC database procedure prohibited with
  the exception of temperature corrections to minimum
  altitudes. \extref{EOMA 8.4.5.2}
  \item[$\bullet$] Fly a stabilised approach if using FPA. Set and pull FPA
    0.3nm before descent point.
\item[$\bullet$] Chart vs. database:
  \iitem Max vertical path difference (fully managed only)\dotfill 0.1°
\item[$\bullet$] For 3D approaches only, flight directors may remain on for
  landing if the glidepath is correctly coded to the threshold.

\item[$\bullet$] APPR must be armed at least 2nm prior to final descent
  waypoint. It should be armed in the segment between the final descent waypoint
  and previous waypoint unless this segment is too short, in which case arm when
  level at final descent altitude.

\item[$\bullet$] \textbf{RNAV approach:} Go around for:
\iitem $>$¾ index fly up or down (fully managed only)
\iitem XTK error > 0.3nm
\iitem NAV ACCUR DOWNGRAD reported by both FMGCs
\iitem FM/GPS POS DISAGREE ECAM
\iitem GPS PRIMARY LOST reported by both FMGCs.
\item If GPS PRIMARY LOST or NAV ACCUR DOWNGRAD is reported by a single
  FMGC continue with the unaffected FMGC.
\item[$\bullet$] \textbf{Overlay approach:} FINAL APP may continue to be
  used unless raw data indicates flight path deviation.
}
\footnotetext{LIDO chart update is in progress. May be titled RNAV(RNP) until completed.}

%% ** SLS approach
\mysec{SLS approach}{
\item[$\bullet$] Check:
\iitem $\bullet$ Aircraft SLS equipped\footnotemark
\iitem $\bullet$ FMA approach capability APPR1
\iitem $\bullet$ SLS approach plate available\footnotemark
\iitem $\bullet$ Associated LPV-LP (SLS) available on ARRIVAL page
\item[$\bullet$] LPV minimas not yet authorised -- use LNAV/VNAV
\item[$\bullet$] FMS fallback not yet authorised -- do not deselect SLS on RADIO NAV page
\item[$\bullet$] Engage both autopilots. Disconnect autopilot by 160ft AGL }
\footnotetext{OFP attachment or FCOM Aircraft Configuration Summary. QRH
  Aircraft Configuration planned Feb 2023.}
\footnotetext{EGNOS is approach procedure box, LPV minimas
available.}

%% ** ILS approach
\mysec{ILS approach}{
\item Standard coverage \extref{LIDO GEN.1.6.2.1}:
  \iitem Localiser ±10° \dotfill 25nm (FAA 18nm)
  \iitem Localiser ±35° \dotfill 17nm (FAA 10nm)
  \iitem Glideslope ±8° \dotfill 10nm
\item Approach Ban \extref{EOMA 8.4.3} \dotfill 1000ft aal  or DA if greater
\item Absolute minimas (DH/TDZ RVR) \extref{EOMA 8.1.3.3.1}:
\iitem Cat I \dotfill 200ft/550m
\iitem LTS Cat I \dotfill 200ft/400m
\iitem Cat II \dotfill 100ft/300m
\iitem Cat IIIA \dotfill $<$100ft/200m
\iitem Cat IIIB \dotfill 0ft/75m
\iitem $\bullet$ Multiple RVRs not required for Cat I / LTS Cat I.
\iitem $\bullet$ Only relevant mid-point or stop end RVRs need to be
  accounted for.
\iitem $\bullet$ Required stop end RVR is always 75m.
\iitem $\bullet$  Required mid point RVR is also 75m if rollout is used,
else it is 125m.
\item LIDO ``Company'' minimas \extref{EOMA 8.1.5.3}
  \iitem Cat IIIB \dotfill No DH
  \iitem Cat IIIA \dotfill 50ft RA
\item Required visual references: \extref{EOMA 8.4.6/10/11}
\iitem Cat I \dotfill Elements of ALS, PAPIS or THR/TDZ
markings/lights
\iitem Cat I LTS, Cat II \dotfill 3 consecutive lights plus a lateral
element
\iitem Cat IIIA \dotfill 3 consecutive lights
\iitem Cat IIIB with DH \dotfill 1 centre-line light
\iitem Cat IIIB No DH \dotfill None
\item FALS/IALS/BALS/NALS \dotfill LIDO GEN 1.5.7.18.6
}\label{ils}


%% ** LVO takeoff
\pagebreak
\mysec[\secref{EOMA 8.1.3.3}]{LVO takeoff}{
\item[$\bullet$] Absolute minima \dotfill 125/125/125
\item[$\bullet$] Reported RVR of initial part of Takeoff run can be replaced by
  pilot assessment.
\item[$\bullet$] Only relevant RVRs must be considered.
\item[$\bullet$] All passenger PEDS must be turned off. \extref{EOMA 8.3.21}
}


%% ** LVO Landing
\mysec[\secref{EOMB 2.3.18.3.1, QRH SI.20}]{LVO landing}{
\item[$\bullet$] All passenger PEDS must be turned off \extref{EOMA 8.3.21}
\item[$\bullet$] Check NOTAMS for airport facility downgrades (see LIDO 1.5.7.16.2)
\item[$\bullet$] Check LVPs in force
\item[$\bullet$] Check aircraft capability (Status page, ADDs, QRH
  Operational Data)
\item[$\bullet$] Check minimas (including downgrades)  and approach ban
\item[$\bullet$] Check autoland wind limits (see page
  \pageref{autoland_wind_limits})
\item[$\bullet$] Check runway condition limits (see page
  \pageref{other_autoland_limits})
\item[$\bullet$] Failure strategies
  \iitem >1000ft aal\dotfill Resolve by 1000ft, inc.\ amend DH $\Rightarrow$ can continue
  \iitem <1000ft, >AH\dotfill Downgrade of landing capability $\Rightarrow$ go-around
  \iitem 350ft\dotfill Incorrect selected course $\Rightarrow$ go-around
  \iitem <200ft\dotfill Autoland warning light $\Rightarrow$ go-around
  \iitem <AH\dotfill Only go around for autoland warning light
  \iitem 30ft\dotfill FLARE and/or THR IDLE not annunciated $\Rightarrow$ go-around
\item[$\bullet$] Extra calls:
\iitem 350ft, FMA:LAND \dotfill PF:``Land'', PNF:``Checked''
\iitem 100ft (only with no DH) \dotfill PNF: ``100'', PF: ``Continue''
\iitem 40ft, FMA:FLARE\dotfill PNF:``Flare''
\iitem 10ft \dotfill Auto callout: ``Retard''
\iitem FMA:ROLLOUT\dotfill PNF:``Rollout''

\item[$\bullet$] Select reverse at touchdown
\item[$\bullet$] Disconnect autopilot at taxi speed
\item[$\bullet$] Notes:
  \iitem Data Lock\dotfill <700ft RA

  \iitem\quad\quad (changes to V$_\textrm{app}$, Wind, Course, and ILS Freq inhibited.)

  \iitem No action on FCU will disengage LAND mode }

%% * END
\pagebreak
\vspace*{5cm}

\begin{center}
  Intentionally blank
\end{center}
\end{document}
